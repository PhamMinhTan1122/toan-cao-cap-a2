\section{\S 2: Ma trận}
\subsection{Định nghĩa}
Một ma trận, đặt tên A (hay, B, C,..) \\
Có kích cỡ \textbf{$m\times n$} gồm \textbf{m dòng} và \textbf{n cột} là bảng hình chữ nhật sau:
$
\begin{bmatrix}
	a_{11} & a_{12} & \cdots & a_{1n} \\
	a_{21} & a_{22} & \cdots & a_{2n} \\
	a_{m1} & a_{m2} & \cdots & a_{mn}
\end{bmatrix} =[a_{ij}]_{m\times n}
$ \\
* Phần tử $a_ij$ nằm tại dòng i và cột j của ma trận A. \\
* Khi số dòng = số cột $(m=n)$ thì A gọi là ma trận \textbf{\underline{vuông cấp n}}.

\subsection{Một số ma trận có dạng đặc biệt}
\begin{enumerate}[label=\arabic{enumi}$^0$.,leftmargin=*]
	\item Ma trận dòng \textbf{thì chỉ có 1 dòng}: \\
	$\begin{bmatrix}
		2 & 4 & 1 & 7
	\end{bmatrix}_{1 \times 4}$
	\item Ma trận cột \textbf{chỉ có 1 cột}: \\
	$
	\begin{bmatrix}
		3 \\
		-1 \\
		4 \\
		2 \\
	\end{bmatrix}_{1 \times 4} 
	$
	\item Ma trận không ($\mathbb{0}$) có mọi phần tử là số 0:
	$
	\begin{bmatrix}
		0 & 0 \\
		0 & 0 \\
	\end{bmatrix}_{2 \times 2},
	\begin{bmatrix}
		0 & 0 \\
	\end{bmatrix}_{1 \times 2}
	$
	\item Ma trận đường chéo \textbf{thì vuông} \\
	$
	\begin{bmatrix}
		a_{11} &  &  & &\\
		& a_{22} &  & \bigzero &\\
		&  & a_{33} & &\\
		&  \bigzero & & \ddots & \\
		&  & & & a_{nn} \\
	\end{bmatrix}_{n \times n}$
	\item Ma trận đơn vị \textbf{(kí hiệu I)} thì vuông và \textbf{trên đường chéo chính toàn số 1}: \\
	+ Cấp 2: $I = \begin{bmatrix}
		\mathbf{1} & 0 \\
		0 & \mathbf{1}
	\end{bmatrix}_{2 \times 2} = I_2$\\
	+ Cấp 3: $I = \begin{bmatrix}
		\mathbf{1} & 0 & 0\\
		0 & \mathbf{1} & 0\\
		0 & 0 & \mathbf{1}
	\end{bmatrix}_{3 \times 3} = I_3$
	\item Ma trận chuyển vị: Cho ma trận $A=[a_{ij}]_{m \times n}$ tùy ý. Nếu đổi (viết) dòng của A thành cột tương ứng sẽ thu được ma trận chuyển vị của A. Kí hiệu $A^\intercal$ \\
	Gặp $\begin{bmatrix}
		a & b & c \\
	\end{bmatrix}_{1 \times 3}
	\Rightarrow A^\intercal = \begin{bmatrix}
		a \\
		b \\
		c \\
	\end{bmatrix}_{3 \times 1}$ \\
	
	* Tính chất \\
	$(A + B)^\intercal =A^\intercal + B^\intercal \\
	(A^\intercal)^\intercal = A \\
	(A.B)^\intercal =  \left\{\begin{matrix}
		A^\intercal.B^\intercal \text{ sai} \\
		B^\intercal.A^\intercal \text{ đúng}
	\end{matrix}\right. $
\end{enumerate}

\subsection{Các phép toán trên ma trận}
\begin{enumerate}
	\item So sánh bằng \\
	Hai ma trận $A=[a_{ij}]_{m\times n}$, $B=[b_{ij}]_{m\times n}$. Bằng nhau viết: $A=B \Leftrightarrow \left\{\begin{matrix}
		\text{cũng cỡ m x} \\
		a_{ij} = b_{ij}, \forall i,j
	\end{matrix}\right. $
	\item Phép cộng (trừ) 2 ma trận cùng cỡ: \\
	*Tổng $A + B = [a_{ij} + b_{ij}]_{m\times n}$ \\
	*Hiệu $A - B = [a_{ij} - b_{ij}]_{m\times n}$ \\
	*Tính chất \\
	! A + B = B + A giao hóa \\
	! A + B + C = (A + B) + C kết hợp \\
	! A + $\mathbb{0}$ = A
	\item Phép nhân 1 số với ma trận \\
	Tích: $\alpha.A = [\alpha.a_{ij}]_{m \times n}$ \\
	*Chú ý: Không có phép toán (1 số) $\pm$ (Ma trận) \\
	*Tính chất \\
	! $(\alpha\beta).A = \alpha(A_{\beta}) = \beta(\alpha A)$ \\
	! $\alpha(A+B) = \alpha A + \alpha B \\
	! $ 0.A = $\mathbb{0}$; (-1).A = -A; A+(-A) = $\mathbb{0}$ \\
	\item Phép toán hai ma trận \\
	Cho $A = [a_{ij}]_{m \times \underline{n}}; B=[b_{ij}]_{\underline{n} \times p}$ \\
	Khi đó, tích: $A.B=[C_{ij}]_{m \times p}$ \\
	Trong đó phần tử $C_{ij}$ nằm tại dòng i và cột j tính bởi: \\
	$C_{ij} = [\text{dòng i của A}] \times [\text{cột j của B}]$ \\
	$C_{ij} = $
	 \begin{tikzpicture}[baseline=(A.center), >=Stealth, every node/.style={inner sep=1pt, font=\small}]
	 	
	 	% Ma trận A (vector hàng)
	 	\matrix (A) [matrix of math nodes,
	 	left delimiter={[},
	 	right delimiter={]},
	 	row sep=1em] {
	 		a_{i1} & a_{i2} & \cdots & a_{in} \\
	 	};
	 	
	 	% Ma trận B (vector cột), đặt phía dưới ma trận A
	 	\matrix (B) [matrix of math nodes,
	 	left delimiter={[},
	 	right delimiter={]},
	 	right=1cm of A] {
	 		b_{1j} \\ 
	 		b_{2j} \\ 
	 		\vdots \\ 
	 		b_{nj} \\
	 	};
	 	
	 	% Vẽ đường cong mũi tên từ a_{i1} ở A đến b_{1j} ở B
	 	\draw[->, black, thick, bend left=45] (A-1-1) to node[above] {$\times$} (B-1-1);
	 	\draw[->, black, thick, bend left=45] (A-1-2) to node[above] {$\times$} (B-2-1);
	 	\draw[->, black, thick, bend right=45] (A-1-4) to node[above] {$\times$} (B-4-1);
	 	
	 	\end{tikzpicture}
	 	$=a_{i1}b_{1j} + a_{i2}b_{2j} +...+ a_{in}b_{nj}$ \\
	 	!Cần nhớ: \\
	 	\begin{itemize}
	 		\item Muốn $A.B$ (A bên trái B) thì điều kiện số cột của A phải bằng số dòng của B.
	 		\item Nói chung kết quả $A.B \neq B.A$
	 	\end{itemize}
	 	!Tính chất \\
	 	\begin{itemize}
	 		\item $A.B.C = (A.B).C $ kết hợp.
	 		\item $A.(B+C) = A.B + A.C $ phân phối.
	 		\item $A.I = I.A =A$.
	 		\item Lũy thừa của ma trận vuông \\
	 		Ta cố $A^1=A, A^2=A.A, A^3=A^2.A,\cdots$ \\
	 		$\boxed{A^k = A^{k-1} \times A}$ \hspace{.5cm} $1 \le k \in \mathbb{N}$ \\
	 		*Nhớ $A^0 = I$. Với I ma trận đơn vị cùng cấp A
	 		
	 	\end{itemize}
	 	
\end{enumerate}