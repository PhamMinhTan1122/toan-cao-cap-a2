\section{\S 1: Số phức}
\subsection{Dạng đại số của số phức}
\fbox{z=a+bi}
\subsection{Dạng lượng giác của số phức}
\boxed {z = r(\cos \varphi + i\sin \varphi)}
\\
Trong đó: \\
- $r = |z| = \sqrt{a^2+b^2}$ module của $z$ \\
- Góc $\varphi$ với $-\pi \le \varphi \le \pi$ gọi là argument của số phức $z$ \\
Ta có \\
$\left\{
\begin{aligned}
	a &= r \cos{\varphi} \\
	b &= r \sin{\varphi}
\end{aligned}
\right.$
$\Longleftrightarrow \left\{
\begin{aligned}
	\cos{\varphi} &= \frac{a}{r} = \dfrac{a}{\sqrt{a^2+b^2}} \\
	\sin{\varphi} &= \frac{b}{r} = \dfrac{b}{\sqrt{a^2+b^2}}
\end{aligned}
\right.$ \\
Giải hệ tìm góc $\varphi$ \\
hoặc
$\tan \varphi = \frac{b}{a} \leftrightarrow \varphi = \arctan \frac{b}{a} (a < 0)$
\subsection{Các phép toán trên dạng lượng giác (NHỚ)}
Phép nhân:\\
\boxed{z_1.z_2 = r_1.r_2 \left[ \cos (\varphi_1 + \varphi_2) +i \sin(\varphi_1 + \varphi_2) \right]} \\
Phép chia: \\
\boxed{\frac{z_1}{z_2} = \frac{r_1}{r_2} \left[\cos(\varphi_1 -\varphi_2) + i\sin(\varphi_1 -\varphi_2) \right]}
\subsection{Công thức Moivre}
Lũy thừa số phức dạng lượng giác \\
\boxed{\begin{aligned}
		z^n &= \left[r (\cos \varphi + i \sin \varphi)\right]^n \\
		&=r^n (\cos\varphi +i\sin\varphi)^n \\
		z^n &= r^n\left[ \cos(n\varphi) +i\sin(n\varphi)\right] 
\end{aligned}}